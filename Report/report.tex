% Options for packages loaded elsewhere
\PassOptionsToPackage{unicode}{hyperref}
\PassOptionsToPackage{hyphens}{url}
\PassOptionsToPackage{dvipsnames,svgnames,x11names}{xcolor}
%
\documentclass[
  letterpaper,
  DIV=11,
  numbers=noendperiod]{scrartcl}

\usepackage{amsmath,amssymb}
\usepackage{iftex}
\ifPDFTeX
  \usepackage[T1]{fontenc}
  \usepackage[utf8]{inputenc}
  \usepackage{textcomp} % provide euro and other symbols
\else % if luatex or xetex
  \usepackage{unicode-math}
  \defaultfontfeatures{Scale=MatchLowercase}
  \defaultfontfeatures[\rmfamily]{Ligatures=TeX,Scale=1}
\fi
\usepackage{lmodern}
\ifPDFTeX\else  
    % xetex/luatex font selection
\fi
% Use upquote if available, for straight quotes in verbatim environments
\IfFileExists{upquote.sty}{\usepackage{upquote}}{}
\IfFileExists{microtype.sty}{% use microtype if available
  \usepackage[]{microtype}
  \UseMicrotypeSet[protrusion]{basicmath} % disable protrusion for tt fonts
}{}
\makeatletter
\@ifundefined{KOMAClassName}{% if non-KOMA class
  \IfFileExists{parskip.sty}{%
    \usepackage{parskip}
  }{% else
    \setlength{\parindent}{0pt}
    \setlength{\parskip}{6pt plus 2pt minus 1pt}}
}{% if KOMA class
  \KOMAoptions{parskip=half}}
\makeatother
\usepackage{xcolor}
\setlength{\emergencystretch}{3em} % prevent overfull lines
\setcounter{secnumdepth}{-\maxdimen} % remove section numbering
% Make \paragraph and \subparagraph free-standing
\makeatletter
\ifx\paragraph\undefined\else
  \let\oldparagraph\paragraph
  \renewcommand{\paragraph}{
    \@ifstar
      \xxxParagraphStar
      \xxxParagraphNoStar
  }
  \newcommand{\xxxParagraphStar}[1]{\oldparagraph*{#1}\mbox{}}
  \newcommand{\xxxParagraphNoStar}[1]{\oldparagraph{#1}\mbox{}}
\fi
\ifx\subparagraph\undefined\else
  \let\oldsubparagraph\subparagraph
  \renewcommand{\subparagraph}{
    \@ifstar
      \xxxSubParagraphStar
      \xxxSubParagraphNoStar
  }
  \newcommand{\xxxSubParagraphStar}[1]{\oldsubparagraph*{#1}\mbox{}}
  \newcommand{\xxxSubParagraphNoStar}[1]{\oldsubparagraph{#1}\mbox{}}
\fi
\makeatother


\providecommand{\tightlist}{%
  \setlength{\itemsep}{0pt}\setlength{\parskip}{0pt}}\usepackage{longtable,booktabs,array}
\usepackage{calc} % for calculating minipage widths
% Correct order of tables after \paragraph or \subparagraph
\usepackage{etoolbox}
\makeatletter
\patchcmd\longtable{\par}{\if@noskipsec\mbox{}\fi\par}{}{}
\makeatother
% Allow footnotes in longtable head/foot
\IfFileExists{footnotehyper.sty}{\usepackage{footnotehyper}}{\usepackage{footnote}}
\makesavenoteenv{longtable}
\usepackage{graphicx}
\makeatletter
\def\maxwidth{\ifdim\Gin@nat@width>\linewidth\linewidth\else\Gin@nat@width\fi}
\def\maxheight{\ifdim\Gin@nat@height>\textheight\textheight\else\Gin@nat@height\fi}
\makeatother
% Scale images if necessary, so that they will not overflow the page
% margins by default, and it is still possible to overwrite the defaults
% using explicit options in \includegraphics[width, height, ...]{}
\setkeys{Gin}{width=\maxwidth,height=\maxheight,keepaspectratio}
% Set default figure placement to htbp
\makeatletter
\def\fps@figure{htbp}
\makeatother

\KOMAoption{captions}{tableheading}
\makeatletter
\@ifpackageloaded{caption}{}{\usepackage{caption}}
\AtBeginDocument{%
\ifdefined\contentsname
  \renewcommand*\contentsname{Table of contents}
\else
  \newcommand\contentsname{Table of contents}
\fi
\ifdefined\listfigurename
  \renewcommand*\listfigurename{List of Figures}
\else
  \newcommand\listfigurename{List of Figures}
\fi
\ifdefined\listtablename
  \renewcommand*\listtablename{List of Tables}
\else
  \newcommand\listtablename{List of Tables}
\fi
\ifdefined\figurename
  \renewcommand*\figurename{Figure}
\else
  \newcommand\figurename{Figure}
\fi
\ifdefined\tablename
  \renewcommand*\tablename{Table}
\else
  \newcommand\tablename{Table}
\fi
}
\@ifpackageloaded{float}{}{\usepackage{float}}
\floatstyle{ruled}
\@ifundefined{c@chapter}{\newfloat{codelisting}{h}{lop}}{\newfloat{codelisting}{h}{lop}[chapter]}
\floatname{codelisting}{Listing}
\newcommand*\listoflistings{\listof{codelisting}{List of Listings}}
\makeatother
\makeatletter
\makeatother
\makeatletter
\@ifpackageloaded{caption}{}{\usepackage{caption}}
\@ifpackageloaded{subcaption}{}{\usepackage{subcaption}}
\makeatother

\ifLuaTeX
  \usepackage{selnolig}  % disable illegal ligatures
\fi
\usepackage{bookmark}

\IfFileExists{xurl.sty}{\usepackage{xurl}}{} % add URL line breaks if available
\urlstyle{same} % disable monospaced font for URLs
\hypersetup{
  pdftitle={Replication Exercise \#1: Report},
  pdfauthor={Priscila Stisman \textbar{} Samuel Cohen},
  colorlinks=true,
  linkcolor={blue},
  filecolor={Maroon},
  citecolor={Blue},
  urlcolor={Blue},
  pdfcreator={LaTeX via pandoc}}


\title{Replication Exercise \#1: Report}
\author{Priscila Stisman \textbar{} Samuel Cohen}
\date{}

\begin{document}
\maketitle

\renewcommand*\contentsname{Table of contents}
{
\hypersetup{linkcolor=}
\setcounter{tocdepth}{3}
\tableofcontents
}

\subsection{Introduction}\label{introduction}

This report will detail our efforts to replicate specific code and
outcomes from Arthur Spirling's 2016 paper ``Democratization and
Linguistic Complexity.'' Spirling's paper explores how the readability
of parliamentary speeches increases overtime, especially as a result of
the Second Reform Act of 1867, which took away property requirements for
voting and enfranchised a significant portion of Britain's population
(Spirling 2016). He hypothesizes that cabinet members' speech
interpretability will become increasingly understandable overtime due to
their prominent roles in government and newfound need to appeal to a new
electorate--- one that was less wealthy, less literate, and less
educated as a whole (Ibid). Contrarily, backbenchers would not need to
change their speech, as they are generally considered the ``rank and
file'' and are not given the same level of public attention (Ibid).
Spirling uses temporal trends of readability metrics (FRE scores, for
instance) and multivariate regressions to assist in his findings, which
were relatively similar to his hypothesis. Our project attempts to
replicate significant portions of this study, specifically temporal
trends of readability, and the primary multivariate regression Spirling
runs. We also add original research to this domain using the same data:
readability of speeches by party, and TF-IDF and cosine similarity score
analysis.

\subsection{Differences \& Similarities}\label{differences-similarities}

\subsubsection{FRE scores}\label{fre-scores}

The FRE Statistics results are very similar, though not identical. Our
values for the minimum, first quartile, median, mean, and third quartile
closely match those in the paper, with only a slight difference in the
third quartile. However, the maximum differs significantly: while the
paper reports a maximum FRE of 205.80, our result is 121.22. The bulk of
the distribution is between 0 and 100, as in the paper.

The average readability score, in both the paper and our replication
exercise, indicates that around the year 1860, the average cabinet
speech becomes more comprehensible than the average non-cabinet speech,
whereas before that, their mean comprehension scores were quite similar.

\subsubsection{FRE Trends}\label{fre-trends}

Spirling plots the mean FRE scores over time by cabinet position (Ibid,
128). The general trend is stable and relatively unchanging before 1867,
after which the scores increase dramatically for cabinet members, and
only slightly for backbenchers (Ibid). Likewise, in our replication, we
get a similar result. One slight difference is that the point of
convergence between cabinet members and backbenchers in terms of
readability happens much earlier than it does in Spirling's plot.
However, this might be due to us aggregating based on year and not by
quarter.~

We also recreate the average syllable count per word score over time
plot (Ibid, 129). We find almost the exact same trends (a decrease for
cabinet MPs, stability for backbenchers), albeit at a smoother rate due
to us aggregating by year and not quarter.~

Overall, the general trends seem to match, with cabinet FRE scores
overtaking those of backbenchers several years before 1867.

\subsubsection{OLS Regression}\label{ols-regression}

The OLS regression results for comprehension scores by cabinet position,
using the same set of controls as the authors, yield very similar
coefficients, though not identical. However, the sign of the
coefficients is consistent in all cases. Some of our coefficients have
lower p-values than those in the paper. For example, the Reform Act
dummy is significant at the 1\% level in our results, while in the
paper, it is significant at the 5\% level.

\subsection{Autopsy}\label{autopsy}

\subsubsection{Data Retrieval}\label{data-retrieval}

We use two primary data sets, which are essentially different iterations
of each other. The first data set is the ``bigframe'' data set used for
quantitative analysis in Spirling's research. This data set contains
metadata on the year a speech took place, the word and syllable count,
FRE score, political party of the MP in question, whether or not they
were a cabinet member, and the competitiveness of their seat. We also
use the raw speech data in both our attempt to recreate much Spirling's
analysis, as well as in our original additions. Like Spirling, we subset
for speeches only between the years 1832 and 1915. We also derive a
random sample of 10,000 from the raw speeches due to the sheer size of
the data.~

Retrieving this data was not very intuitive. While Spirling has a
Harvard Dataverse page for his replication materials, it only included
the code and bigframe numerical data---that is, there was no raw data in
this repository. To find the raw speech data, we navigated to the Arthur
Spirling and Andy Eggers Database, which included the CSVs for the raw
data.

\subsubsection{Replication Challenges}\label{replication-challenges}

One of the primary challenges we faced was the fact that we used only a
sample of the raw speech data. While we believed 10,000 observations out
of well over 600,000 would be representative, there were still some
discrepancies in our results. For instance, while the mean and median
FRE scores for our sampled speeches were relatively similar to those in
the paper, the standard deviation was a lot larger, and the minimum and
maximum values we quite different as well.

Another challenge we encountered was interpreting very large and very
small FRE scores. While FRE scores typically range between 0 and 100,
both the paper and our replication indicated scores much higher and
lower than these thresholds. After some research, we discovered that it
is indeed possible to have FRE scores below 0 (if language is
particularly complex).

\subsubsection{Replication Successes}\label{replication-successes}

Perhaps our greatest replication success was our results for Spirling's
multivariate regression. Coefficients differed only slightly, if at all.
This success was likely due to us using the bigframe data rather than
our samples and cleaned raw data. While a LaTex illustration of the
regression was not provided in the paper, we were still able to recreate
it with good results.

\subsection{Extension}\label{extension}

\subsubsection{Party Affiliation and Speech
Readability}\label{party-affiliation-and-speech-readability}

For our original additions, we decided to look at how readability
evolves along party lines. The majority of Spirling's paper focuses on
how cabinet status affects speech readability, rather than party
affiliation. We decided to explore speech readability over time by
political party.

We graphed the mean FRE scores by year over time, and then disaggregated
the trend lines by political party: Conservative and Liberal. We
hypothesized that conservative MPs would likely not change their speech
as much as Liberals, due to our belief that Liberals would garner more
working class support.

Our findings indicate that speech becomes easier over time after 1867
for all MPs, regardless of party affiliation. However, interestingly,
after 1867, as one party increases their speech readability over a short
time, the other often decreases, and vice versa.

\subsubsection{TF-IDF Weighting}\label{tf-idf-weighting}

For our original analysis, we also used the raw speech data for text
analysis. Spirling's analysis mostly focuses on readability, so we
decided to explore a bit more in the weeds. We first tokenized,
preprocessed, and created a data frequency matrix (DFM). We then took
the weighted TF-IDF scores for each token to find which words carried
the most importance throughout the corpus. We find that simple
procedural words have the highest scores, as speeches can often be very
formulaic. For instance, ``yes,'' ``sir,'' ``amend,'' ``order,'' and
``bill'' had the highest weighted scores.

\subsubsection{Cosine Similarity}\label{cosine-similarity}

Finally, we used cosine similarity to find the speeches that are most
similar with each other. We utilize our TF-IDF matrix to find the
closest cosine similarity for each document. Similar to the highest
TF-IDF scores, the speeches with the highest similarity scores exhibited
very formulaic language that appears to be common in parliamentary
procedure. For instance, the speech duo with the highest cosine
similarity score between each other was an MP in 1876 saying ``he will
repeat the question on Monday,'' and an MP in 1890 saying ``I will
repeat the question on Monday.'' The next highest duo was an MP in 1872
saying ``He would withdraw the amendment,'' and another in 1899 saying
``I will withdraw the amendment.''

\subsubsection{Suggested Improvements}\label{suggested-improvements}

\begin{itemize}
\item
  Easier to access raw data: One of our primary challenges was finding
  the raw speech data Spirling used in his analysis. Adding this to the
  Harvard Dataverse repository would be a welcome update, and could help
  future researchers replicate and innovate at a much easier rate.
\item
  Party affiliation analysis: While we provide a preliminary analysis on
  readability and party affiliation, additional research could help
  determine how political parties develop their speech over time. The
  British Political Manifestos corpus could be of significant use here.
\item
  More analysis of raw text: TF-IDF, cosine similarity, KWIC, etc.:
  Future research should focus more on analysis of the text in speeches,
  not just readability, as interesting and insightful details may get
  lost in the weeds. In addition, sentiment analysis would also be a
  welcome addition-- it would be interesting to see how MPs in different
  cabinet positions or political parties change not only their speech,
  but the tone of their message over time in response to various
  extraneous factors.
\item
  Comparing different readability scores: One final action future
  researchers can take is to compare FRE scores to other metrics of
  readability, such as Dale Chall or SMOG. This would add robustness and
  would make it easy to spot any discrepancies.
\end{itemize}

\subsection{References}\label{references}

Eggers, A. \& Spirling A. (Accessed 2025). Eggers and Spirling Database.
\url{https://andy.egge.rs/eggers_spirling_database.html}

Prokopets, M. (Accessed 2025). The Beginner's Guide to Flesch Reading
Ease Scores. Nira. https://nira.com/flesch-reading-ease/

Replication Materials for: `Democratization and Linguistic Complexity:
The Effect of Franchise Extension on Parliamentary Discourse,
1832--1915' - The Journal of Politics. (Accessed 2025). Harvard
Dataverse.
https://dataverse.harvard.edu/dataset.xhtml?persistentId=doi:10.7910/DVN/DDQDMS

Spirling, A. (2016). Democratization and Linguistic Complexity: The
Effect of Franchise Extension on Parliamentary Discource, 1832-1915.
Journal of Politics. 78(1).




\end{document}
